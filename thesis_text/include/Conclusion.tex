% CREATED BY DAVID FRISK, 2016
\chapter{Conclusion}\label{chapter:conclusion}
This thesis has successfully formalized a significant amount of the theory of \textbf{Poly} and implemented several dynamical systems relying on the theory. 
The theory chapters have formalized the categories themselves, initial objects, terminal objects, products, coproducts, composition and parallel product as monoidal structures, exponential object, charts, quadruple adjunction with sets, as well as many more various proofs, examples, and lemmas needed. 
The implementation part has shown how lenses can represent dynamics in dynamical systems, how polynomials act as interfaces, how systems are wired together, how to implement systems, how to install behavior into wiring diagrams, and how to run systems arriving at concrete programs.
Lenses between special polynomials have been shown to correspond to different concepts, such as DFA's or Moore machines, which shows the power polynomials and lenses possess as an abstraction generalization.
Some programs implemented are Fibonacci sequence generators, the Lorenz system, the Hodgkin-Huxley model, and reservoir computers.
Combining theory and applications, commuting squares between lenses and charts have been used to show how one dynamical system can simulate another.

The work of this thesis has met its purpose of contributing to the polynomial functors ecosystem, showing the feasibility of formalizing \textbf{Poly} and associated categories, as well as providing a solid ground to build up more advanced concepts, both theoretical and practical. 
