% CREATED BY DAVID FRISK, 2016
\chapter{Conclusion}\label{chapter:conclusion}
This thesis has successfully succeeded in both formalizing a significant amount of theory of \textbf{Poly} as well as implementing dynamical systems using the category while relying heavily on the theory. The theory chapters have formalized the categories themselves, initial objects, terminal objects, products, coproducts, composition and parallel product as monoidal structures, exponential object, charts, quadruple adjunction with sets, as well as many more various proofs, examples, and lemmas needed. 
 The implementation part has shown how arrows can be seen as dynamics in dynamical systems, how polynomials act as interfaces, how to wire systems together, how to implement systems, how to install behaviour into wiring diagrams, and how to run systems arriving at concrete programs. How arrows between correspond to different concepts such as DFA's or Moore machines depending on the polynomials of the arrow shows the power polynomials and lenses as an abstract generalization posses. Some of the programs implemented are Fibonacci sequence generators, Turing machines, the Lorenz system, the Hodgkin-Huxley model and reservoir computers. Combining theory and applications, commuting squares between lenses and charts have been used to show how one dynamical system can simulate another for two different examples.

The work of this thesis has succeeded in contributing to the polynomial functors ecosystem, showing the feasibility of formalizing \textbf{Poly} and associated categories, as well providing a solid ground to build up more advanced concepts, both theoretical and practical. 
