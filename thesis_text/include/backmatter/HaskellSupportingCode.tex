% CREATED BY DAVID FRISK, 2016
\chapter{Appendix 1: Haskell supporting code}
\label{app:haskell}

The Agda FFI is used to interact with Haskell and its ecosystem for three main purposes: fast matrix inversion, plotting graphs, and a convenient command-line interface for running and plotting the continuous dynamical systems. The Nix package manager \cite{nix} is used to plug the Haskell code and its ecosystem with the Agda code.

\section{Matrix inversion}
In the reservoir example \ref{sec:reservoir}, many matrix operations are used. Many of these operations are implemented directly in Agda - see the \texttt{Dynamical/Reservoir/Matrix} folder - using facilities from the standard library's \texttt{Data.Vec} type. However, inverting a matrix is an expensive operation that requires an algorithm that is quite performance-intensive. For this reason, an existing implementation in the Haskell ecosystem from the HMatrix \cite{hmatrix} library is used. This is done by declaring a postulate and using the \texttt{FOREIGN} pragma in the Agda module, and accessing the Haskell module by importing it in the generated Haskell code:
\begin{minted}{agda}
...
{-# FOREIGN GHC
import HsMatrix
#-}

postulate
  invertMatrixAsListTrusted : List (List ℝ) → List (List ℝ)
{-# COMPILE GHC invertMatrixAsListTrusted = invertMatrixAsList #-}
...
\end{minted}
The matrix is temporarily represented as a list of lists, where each list is a row, to simplify translating the matrix dimensions to the Agda level. In this way, only the type conversions directly in Agda have to be taken care of. The code in the Haskell module is simply this:
\begin{minted}{haskell}
...
import Numeric.LinearAlgebra qualified as HMat
...
invertMatrixAsList :: [[Double]] -> [[Double]]
invertMatrixAsList = HMat.toLists . HMat.pinv . HMat.fromLists
\end{minted}

The matrix is then converted back to the Agda representation - a length-indexed vector of length-indexed vectors, by using the \texttt{trustMe} construct from the standard library to convince Agda that the matrix has the right dimensions:

\begin{minted}[escapeinside=||]{agda}
...
open import Relation.Binary.PropositionalEquality.TrustMe
...
fl : ∀ {A : Set} {n} → (l : List A) → {l : L.length l ≡ n} → Vec A n
fl l {refl} = fromList l

fromListOfLists : ∀ {n m} → 
    (l : List (List ℝ)) → 
    {p₁ : L.length l ≡ n} → 
    {p₂ : M.map L.length (L.head l) ≡ M.just m}
    → Matrix ℝ n m
fromListOfLists [] {refl} = |$\mathbb{M}$| []
fromListOfLists (x ∷ xs) {refl} {p} = 
    |$\mathbb{M}$| (fl x {maybepr p} ∷ 
            (V.map (\x → fl x {trustMe}) $ fl xs {trustMe}))
    where maybepr : ∀ {m n} → M.just m ≡ M.just n → m ≡ n
          maybepr refl = refl

_⁻¹ : ∀ {n : ℕ} → Matrix ℝ n n → Matrix ℝ n n
_⁻¹ {n} (|$\mathbb{M}$| m) =
  let asList = toList $ V.map toList m
      inverted = invertMatrixAsListTrusted asList
  in fromListOfLists inverted {trustMe} {trustMe}
infixl 40 _⁻¹
\end{minted}

In this way, the \texttt{\_⁻¹} operation can be made to take in square matrices and return square matrices of the same dimension.

\section{Graph plotting}

The figures in the continuous systems are all generated via the Haskell library Chart \cite{chart}. This case is similar to the above, with the simplification that there is no need to use \texttt{trustMe}. Instead, the Agda postulate for the plotting function runs in IO. A minor issue is solved by implementing a custom product type to guarantee the FFI that no dependent typing is involved. The postulate code is as follows:
\begin{minted}{agda}
open import Data.Product as P hiding (_×_) renaming (_,_ to _,p_)
record _×_ (A B : Set) : Set where
  constructor _,_
  field
    fst : A
    snd : B

fromSigma : {A B : Set} → A P.× B → A × B
fromSigma ( a ,p b ) = a , b

postulate
  plotDynamics : String → 
                 String → 
                 String → 
                 Float → 
                 List (String × List Float) → IO ⊤

{-# FOREIGN GHC import HsPlot #-}
{-# COMPILE GHC plotDynamics = plotToFile #-}
{-# COMPILE GHC _×_ = data (,) ((,)) #-}
\end{minted}

The corresponding Haskell code is small enough that the entire module is included:

\begin{minted}{haskell}
-- Dynamical/Plot/src/HsPlot.hs
{-# LANGUAGE BlockArguments #-}
{-# LANGUAGE ImportQualifiedPost #-}
{-# LANGUAGE OverloadedStrings #-}

module HsPlot where

import Graphics.Rendering.Chart.Easy
import Graphics.Rendering.Chart.Backend.Cairo
import Control.Monad (forM_)
import Data.Text qualified as T

plotToFile :: T.Text ->
              T.Text -> 
              T.Text -> 
              Double -> 
              [(T.Text, [Double])] -> 
              IO ()
plotToFile xaxisTitle yaxisTitle title dt lines = 
    toFile def (T.unpack . T.replace " " "_" $ T.toLower title <> ".png") $ 
    do
        layout_title .= T.unpack title
        layout_x_axis . laxis_title .= T.unpack xaxisTitle
        layout_y_axis . laxis_title .= T.unpack yaxisTitle
        setColors . fmap opaque $ 
            [purple, sienna, plum, 
             powderblue, salmon, sandybrown, 
             cornflowerblue, blanchedalmond, 
             firebrick, gainsboro, honeydew]
    forM_ lines \(name, l) ->
        plot (line (T.unpack name) [zip [0, dt..] l ])
\end{minted}

The string arguments are given as \texttt{Data.Text.Text}, even though \texttt{Chart} handles \texttt{String}s, because that is what Agda's \texttt{Data.String.String} corresponds to; see \href{https://agda.readthedocs.io/en/v2.6.2.2/language/foreign-function-interface.html#built-in-types}{the table in the official documentation}.

\section{Command-line interface}
Finally, there is the issue of conveniently running a system of choice, with a given selection of parameters. This is solved by using optparse-applicative \cite{optparse-applicative}, a well-established Haskell library. The FFI code looks similar to above, but the main interest is now to convert between types that are pattern-matched on in the main Agda program. The Agda code, with the type containing the paramaters of the different systems is as follows:
\begin{minted}{agda}

{-# FOREIGN GHC 
import OptparseHs
#-}

data SystemParams : Set where
  ReservoirParams : (rdim : ℕ) → 
                    (trainSteps touchSteps outputLength : ℕ) → 
                    (lorinitx lorinity lorinitz dt variance : Float) → 
                    SystemParams
  LorenzParams    : (lorinitx lorinity lorinitz dt : Float) → SystemParams
  HodgkinHuxleyParams : (dt : Float) → SystemParams
  LotkaVolterraParams : (α β δ γ r0 f0 dt : Float) → SystemParams
{-# COMPILE GHC SystemParams = data SystemParams
   (ReservoirParams | 
    LorenzParams | 
    HodgkinHuxleyParams | 
    LotkaVolterraParams) #-}

record Options : Set where
  constructor mkopt
  field
    system  : SystemParams

{-# COMPILE GHC Options = data Options (Options) #-}

postulate 
  parseOptions : IO Options
{-# COMPILE GHC parseOptions = parseOptions #-} 
\end{minted}

And its corresponding Haskell type is:

\begin{minted}{haskell}
data SystemParams where
  ReservoirParams :: 
    { rdimf         :: Integer
    , trainStepsf   :: Integer
    , touchStepsf   :: Integer
    , outputLengthf :: Integer
    , lorinitx      :: Double
    , lorinity      :: Double
    , lorinitz      :: Double
    , dt            :: Double
    , variance       :: Double
    } -> SystemParams
  LorenzParams ::
    { lorinitx' :: Double
    , lorinity' :: Double
    , lorinitz' :: Double
    , dt'       :: Double
    } -> SystemParams
  HodgkinHuxleyParams ::
    { dt'' :: Double }
    -> SystemParams
  LotkaVolterraParams ::
    { alpha :: Double
    , beta :: Double
    , delta :: Double
    , gamma :: Double
    , r0 :: Double
    , f0 :: Double
    , dt''' :: Double }
    -> SystemParams
deriving instance Show SystemParams
\end{minted}

The code for parsing this type looks like this:
\begin{minted}{haskell}
systemParser :: Parser SystemParams
systemParser = hsubparser $ 
     command "Reservoir" 
        (info reservoirParamsParser (progDesc "Reservoir system"))
  <> command "Lorenz" 
        (info lorenzParamsParser (progDesc "Lorenz system"))
  <> command "HodgkinHuxley" 
        (info hodgkinHuxleyParamsParser (progDesc "Hodgkin-Huxley system"))
  <> command "LotkaVolterra" 
        (info lotkaVolterraParamsParser (progDesc "Lotka-Volterra system"))

reservoirParamsParser :: Parser SystemParams
reservoirParamsParser = ReservoirParams
  <$> option auto 
    (long "numNodes" <> short 'n' <> metavar "NUMNODES" <> 
        help "Number of nodes in the reservoir")
  <*> option auto 
    (long "trainingSteps" <> short 't' <> metavar "TRAINSTEPS" <> 
        help "Number of data points to train on")
  -- ... other params

lorenzParamsParser :: Parser SystemParams
lorenzParamsParser = LorenzParams <$> -- ... params
-- other systems with a similar type signature
\end{minted}

It is worth remarking that Haskell and optparse-applicative allow for a concise expression of more type safety than a simple sum type, through the use of the language extensions for GADTs \cite{ghc-gadts} and DataKinds \cite{ghc-data-kinds}, by having the \texttt{SystemParams} type be parametrized by the kind of a separate type called \texttt{ParamType}:

\begin{minted}{agda}
data ParamType
    = ReservoirParamsType 
    | LorenzParamsType 
    | HodgkinHuxleyParamsType 
    | LotkaVolterraParamsType
  deriving Show
\end{minted}

Which can then be given as an argument to \texttt{SystemParams}:
\begin{minted}{haskell}
data SystemParams (p :: ParamType) where
  ReservoirParams :: 
    { rdimf         :: Integer
    , trainStepsf   :: Integer
    , touchStepsf   :: Integer
    , outputLengthf :: Integer
    , lorinitx      :: Double
    , lorinity      :: Double
    , lorinitz      :: Double
    , dt            :: Double
    , variance       :: Double
    } -> SystemParams ReservoirParamsType
    ... other constructors
\end{minted}

This way, the individual parsers can be made to only parse specific sets of parameters. Forgetting an argument would give a compilation error:

\begin{minted}{haskell}
lorenzParamsParser :: Parser (SystemParams LorenzParamsType)
lorenzParamsParser = LorenzParams
  <$> option auto (long "x0" <> short 'x' <> metavar "X0" <> 
    help "Initial x value")
  <*> option auto (long "y0" <> short 'y' <> metavar "Y0" <> 
    help "Initial y value")
  -- <*> option auto (long "z0" <> short 'z' <> metavar "Z0" <> 
  --  help "Initial z value") error! (SystemParams LorenzParamsType)
  -- expects this float
  <*> option auto (long "dt" <> short 't' <> metavar "DT" <> 
    help "Time step")
\end{minted}

This is a preferable solution in the Haskell level, but converting the types given as arguments to a higher-kinded type in Agda to the DataKinds-produced types proved tricky, so we opted for the simpler solution of a pure sum type.

\section{Building}

The challenge was to get the Agda code that uses the Haskell code to have access to the larger Haskell ecosystem in an easily reproducible and user-friendly way. This is important because of the intensity of debugging in programming with polynomial functors - it is regular programming after all, just in a different framework.

\subsection{Nix}

The Nix\cite{nix} package manager is a good solution to this problem. It is widely used by the Haskell community, so it has good support and tooling for it, and provided a reliable way to build across the authors of the thesis, and for anyone wishing to use this code for themselves.

The most important primitive of Nix is the \textit{derivation}. A derivation is an expression that describes how to build, package, install, run and in some cases deploy a piece of software. This can be an executable, a library, or a combination thereof. Derivations are meant to be deterministic, and will build only once unless the source code of the software or the derivation itself change. We tried to keep the Nix code well organized and encapsulating different responsibilities, though there's not much of it.

\subsection{Building Agda with Nix}

As mentioned, a Nix derivation includes a description of how to build a piece of software. In our case, this is the Agda executable that runs and plots the chosen system's dynamics, in \texttt{Dynamical/Plot.agda}. The actual building happens in a derivation's \texttt{buildPhase} attribute, and the dependencies go in \texttt{nativeBuildInputs} or \texttt{buildInputs}, depending on whether they're build or runtime dependencies. Nix provides a lot of flexibility on how to handle different types of depedencies, however, and there are other phases that can handle different parts of the build process, like \texttt{configurePhase}.

Our dependencies are the Agda libraries we use, which are \texttt{agda-categories}, \texttt{cubical} and \texttt{standard-library}, as well as the Haskell packages we have to use for each piece of Haskell code. Haskell has its own tooling regarding package management, the most conventional of which is Cabal, which we use to list the dependencies for each piece of Haskell code. We want to make these two ecosystems interact, and we can do this via the Agda compiler's provided method to give flags to GHC directly, the flag \texttt{--ghc-flag=}.

The question then is: how do we translate Haskell dependencies into the Nix environment, which we can then use to feed into the Agda compiler we invoke in the \texttt{buildPhase}? For this, we use a Nix tool called \texttt{cabal2nix}, which reads a *.cabal file and converts it to a Nix expression that can be manipulated like any other data structure in the language. We then collect these dependencies into a custom GHC environment, also via Nix: this creates a GHC executable that has access, in its own internal package database, to the Haskell packages provided.

There are also some more details to take care of, like overriding the Nix package repository's version of the Cubical library, configuring the build environment for Agda, and providing include flags for GHC. The code for all of this is in the expression below:
\begin{minted}{nix}
// agda.nix
  { pkgs ? import <nixpkgs> { } }:

  let
    agdaDepsNames = [ "standard-library" "agda-categories" "cubical" ];
    agdaDeps = builtins.map (n:
      if n == "cubical" then
        pkgs.agdaPackages.${n}.overrideAttrs (old: {
          version = "0.4";
          src = pkgs.fetchFromGitHub {
            repo = "cubical";
            owner = "agda";
            rev = "v0.4";
            sha256 = "0ca7s8vp8q4a04z5f9v1nx7k43kqxypvdynxcpspjrjpwvkg6wbf";
          };
        })
      else
        pkgs.agdaPackages.${n}) agdaDepsNames;
    getCabalStuff = name: path:
      (builtins.filter (d: !(isNull d))
        (pkgs.haskellPackages.callCabal2nix name path { }).propagatedBuildInputs);
    haskellDeps = (getCabalStuff "HsPlot" ./Dynamical/Plot/HsPlot.cabal);
    customGhc = pkgs.haskellPackages.ghcWithPackages (ps: haskellDeps);
    nativeBuildInputs = with pkgs; [
      customGhc
      (agda.withPackages (p: agdaDeps))
    ];
    commonConfigurePhase = ''
      export AGDA_DIR=$PWD/.agda
      mkdir -p $AGDA_DIR
  
      for dep in ${pkgs.lib.concatStringsSep " " agdaDepsNames}; do
        echo $dep >> $AGDA_DIR/defaults
      done
    '';
    buildCommand = haskellSourcePaths: ''
      agda -c \
        ${
          pkgs.lib.concatMapStrings (path: "--ghc-flag=-i${path} ")
          haskellSourcePaths
        } \
        ${
          pkgs.lib.concatMapStrings (dep: "--ghc-flag=-package=${dep.name} ")
          haskellDeps
        } \
        --ghc-flag=-package-db=${customGhc}/lib/ghc-${customGhc.version}/ \\
          package.conf.d \
        Dynamical/Plot/Plot.agda
    '';
  in { inherit commonConfigurePhase nativeBuildInputs buildCommand; }  
\end{minted}

Not to get into too much detail - this expression is a big \texttt{let}-block that binds the three variables \texttt{commonConfigurePhase}, \texttt{nativeBuildInputs} and \texttt{buildCommand} and exposes them. We then import these two variables in the files \texttt{default.nix} and \texttt{shell.nix}:

\begin{minted}{nix}
  // default.nix
  { pkgs ? import <nixpkgs> { } }:

  let
    agda = import ./agda.nix { inherit pkgs; };
    binName = "plot";
  in pkgs.stdenv.mkDerivation {
    name = binName;
    src = ./.;
    nativeBuildInputs = agda.nativeBuildInputs;
    configurePhase = agda.commonConfigurePhase;
    buildPhase = agda.buildCommand 
      [ "Dynamical/Plot/src" "Dynamical/Matrix/src" ] ;
    installPhase = ''
      mkdir -p $out/bin
      cp $TMP/poly/${binName} $out/bin
    '';
  }  
\end{minted}

This file allows a user to invoke \texttt{nix-build} in the folder it's located to run it as almost a script. It ultimately runs the build command defined in \texttt{agda.nix} and puts it into a \texttt{result} symbolic link that points to the Nix store, where all Nix-evaluated expressions' results live.

There's also the file \texttt{shell.nix}, which is used with the command \texttt{nix-shell} instead. We use to expose a convenient command inside a shell environment that reuses the build artifacts in the current folder. This is unlike running \texttt{nix-build}, which always creates an isolated environment in which to build.

\begin{minted}{nix}
// shell.nix
{ pkgs ? import <nixpkgs> {} }:

let
  agda = import ./agda.nix { inherit pkgs; };
in pkgs.mkShell {
  nativeBuildInputs = agda.nativeBuildInputs;
  configurePhase = agda.commonConfigurePhase;
  shellHook = ''
    export build="${agda.buildCommand 
      [ "Dynamical/Plot/src" "Dynamical/Matrix/src" ]}"
  '';
}
\end{minted}

From within the Nix shell, we can run \texttt{\$build} and get an Agda executable named \texttt{Plot} that runs our program.
