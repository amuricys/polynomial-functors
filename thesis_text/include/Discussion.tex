% CREATED BY DAVID FRISK, 2016
\chapter{Discussion}\label{chapter:discussion}
This chapter covers a discussion on certain topics about the formalization of \textbf{Poly}. The characterization of equality between lenses is discussed, how useful Cubical Agda was in the formalization, as well as the ease of use of \textbf{Poly} in implementing dynamical systems. Further, the consequence of mixing Cubical Agda with \texttt{agda-categories} is discussed, and the notion of wild polynomials. Finally, some areas of future work are given.

\section{Equality of lenses}
The fact that lenses (as well as polynomials and charts) are defined as (dependent) sigma types makes equality of lenses (as well as polynomials and charts) a major pain point. The first component of the sigma type is directly comparable, but for the second component, \mintinline{agda}{subst} is needed to make the types comparable. This problem shows up every time an equality between lenses needs to be proved. If the proof of the first component being equal is \mintinline{agda}{refl}, the \mintinline{agda}{subst} is easily removed by using \mintinline{agda}{substRefl}. But, anytime the proof of the first component is non-trivial, the complexity of the equality proof quickly increases. Therefore, the more powerful variant of lens equality was developed, \mintinline{agda}{lens≡ₚ}, where the \mintinline{agda}{subst} is pushed inside the term as much as possible. This variant of lens equality has shown to be very useful and is used in most places in the code.
Hopefully, the efforts to characterize lens equality are useful to any future formalizations of \textbf{Poly}.

Although the same solution is used for charts, there is a possible alternative approach.
A chart can be represented as a single function, satisfying a predicate instead of being defined as two different maps in a sigma type.
This representation makes equality between charts to be simply equality between functions. Although, this approach was not explored further.

\section{Proving with Cubical Agda}
Cubical Agda provides some very useful features, such as:
\begin{itemize}
    \item Identifying isomorphism and equality,
    \item Transporting equalities between types and vice versa,
    \item Function extensionality,
    \item Wealth of lemmas and properties in its standard library,
    \item Substitution of types via \texttt{subst},
    \item Higher Inductive Types.
\end{itemize}

However, some aspects of it are not as smooth. No pattern matching on \texttt{refl} sometimes complicates proofs, constant transports are not always equal to \texttt{refl} for subtle technical reasons, nested \texttt{subst} and \texttt{transport} are very difficult to deal with, introducing interval variables in proof definitions leads to warnings that make for a noisy IDE experience, and the names of lemmas and their documentation in the standard library are not informative enough. The issue of nested transports is a particularly thorny one: very often, an equality type states something that a human can verify informally quite easily, but doing the needed plumbing to get a proof to compute is difficult and results in extremely long proofs that are actually simple in idea.

\section{Challenges in implementing dynamical systems}
The implementation of a couple of dynamical systems has led us, the authors of the thesis, to an impression of how it is to use polynomial functors in a practical implementation. A few different aspects stand out when it comes to writing dynamical systems in Agda using polynomial functors.

\subsection{Datatypes in Agda}

When creating dynamical systems, many data types need to be created, making it easy to get lost in the details. There need to be data types for the state of each system, for the positions and output of each system, as well as for each input at each output. This is easier to handle on paper by abusing notation and leaving details out, but in Agda, everything is explicit, making it very verbose. Also, using polynomials created by the operators defined, such as addition or parallel product, is annoying. In these cases, the data types used behind the scenes are coproduct and product, with general constructor names that are terrible to work with when implementing dynamical systems, or wiring the wiring diagrams.

There is also the issue of handling the empty type $\bot$ when programming, especially when it is a direction at some position. It is of course impossible to provide a value of this type, so while it in theory nicely represents a state where "no input can be provided", in practice this means that a lot of care and thought need to be given to how to either avoid getting in such a state, or never attempt updating the system when there.

\subsection{Delayed timesteps}
When designing and implementing a dynamical system, it is important to consider timing in the flow of information. For example, given an input to a system, the output for that input is not received until the next time step. This creates a delay, which is in fact used by the Fibonacci example, where the identity system simply delayed the input by one frame. However, taking into account these delays in more extensive systems creates a lot of complexity to keep track of. The reservoir system suffered from this as well, needing state transition counters to account for delays in the inputs. In a setting like Python + NumPy, all the states can always be accessed as indices in a matrix that keeps track of the history if needed.

\subsection{Printing}
When the systems did not work as expected, a lot of effort was put into figuring out what was going wrong. For example, a first attempt at debugging the reservoir system was by manually changing the output type of all systems to include their state, which led to a stream of a massive product type to be analyzed by the caller in \texttt{Plot.agda}. When this became unwieldy, the idea to use the Haskell FFI to call out to \texttt{Debug.Trace.trace} was introduced, but this comes at the expense of some flexibility since it cannot be used in the \texttt{readout} functions, as the \texttt{update} function depends on that function's output. A tool similar to debuggers for imperative languages would have been useful for stepping through the dynamical systems and looking into their internal state, or perhaps some abstraction that plugs into dynamical systems in a nice way that we could not think of.

\subsection{Freezing system states}
Sometimes, we want a system to be unable to change some aspect of its state, but want them to "carry their state around" and transition its contents to some other state. An example of this is the reservoir's \texttt{outputWeights}. Recalling what that datatype looks like:

\begin{minted}{agda}
data ReservoirState (numNodes : ℕ) (systemDim : ℕ) : Set where
  Coll : (nodeStates : Vec ℝ numNodes)
         (counter : ℕ)
         (statesHistory : Vec (Vec ℝ numNodes) counter) 
         (systemHistory : Vec (Vec ℝ systemDim) counter) → 
         ReservoirState numNodes systemDim
  Touch : (nodeStates : Vec ℝ numNodes)
          (counter : ℕ)
          (outputWeights : OutputWeights numNodes systemDim) -- should be
                                                             -- immutable!
          →
          ReservoirState numNodes systemDim
  Go : (nodeStates : Vec ℝ numNodes)
       (outputWeights : OutputWeights numNodes systemDim)    -- should be
                                                             -- immutable!
       →
       ReservoirState numNodes systemDim
\end{minted}

There is no straightforward way, as far as we can tell, to prevent the \texttt{outputWeights} component of the reservoir from being updated in the type level. An attempt was made by wrapping the contents of the \texttt{Touch} and \texttt{Go} states in a \texttt{Reader} monad, and then having the readout function unroll the computation contained in it, but this was quite hard to work with due to how \texttt{Reader} works in Agda - it is designed to be capable of working in categories other than \textbf{Set}, so more general than Haskell's version, but this causes the need for more manual tinkering to get it right. And even if it worked, still nothing would prevent the state update function from simply replacing the computation with something else. Perhaps linear types could be of help here too, by ensuring that particular component of the state is only used once when in the \textit{touching} stage.

\section{Mixing Cubical Agda with \texttt{agda-categories}}
A big consequence of mixing Cubical Agda with \texttt{agda-categories} is that it hinders the formalization from being merged into either of the repositories.
The choice was to use Cubical Agda for its usefulness in proofs and \texttt{agda-categories} for its richness of categorical constructs.
However, Cubical Agda has its own category theory library and would not want any code using \texttt{agda-categories} to be merged into it.
At the same time, \texttt{agda-categories} would not want any code that depends on Cubical Agda. This issue hinders the code from being integrated into any of these repositories.

The code could be modified to be mergeable into the Cubical Agda library. All the categorical constructs used from \texttt{agda-categories} could be rewritten to use Cubical Agda's constructs. The constructs only available in \texttt{agda-categories} could be defined and added to the Cubical library. At the core, all the proofs should remain the same, although they might need some slight modifications. An important detail that must be considered when adapting the code only to use Cubical Agda is the notion of wild polynomials.

\section{Wild polynomials}
Some requirements differ between providing an instance of a category in \texttt{agda-categories} versus in the Cubical Agda library.
One crucial difference is that Cubical Agda requires the lenses to be sets, using the cubical notion \mintinline{agda}{isSet}. To provide a proof that lenses are sets, it is required that the type of positions and all directions of polynomials are sets. This difference gives rise to two different variants of polynomials. The normal polynomials, also called wild polynomials, without any requirement of the positions or directions being sets, are used to define a category in \texttt{agda-categories}. As well as the more strict variant of polynomials, called \mintinline{agda}{SetPolynomial}, used to define an instance of a category in Cubical Agda library. 

Both variants of polynomials are provided in the code, but wild polynomials are mostly used as they are more general. Wild polynomials also avoid the use of \mintinline{agda}{isSet}, which would add noise to the code. However, set polynomials together with an instance of a category for Cubical agda library can be found in the code. In fact, some constructs and proofs require polynomials to be sets, such as the proof showing that \textbf{Poly} has equalizers.

%Definition of category differs between \texttt{agda-categories} and Cubical agda.
% Cubical agda library requires the morphisms to be isSet. While this requirement is not needed for \texttt{agda-categories}. This makes \texttt{agda-categories} variant less strict and more general, and therefore the polynomials are considered "Wild". For cubical agda, to provide an instance, polynomials are more constrained, by forcing the position to be isSet, and for each direction to be isSet. This is needed to construct a proof that lenses between these strict polynomials are isSet. Which is needed to arrive at an instance for the. The same procedure would be needed for the category \textbf{Chart} in Cubical. 

%In SetPoly, we add this strict polynomials and show that it also forms a category in \texttt{agda-categories}. But we also there provide an instance of a category in cubical. The fact that polynomials are sets, more specifcally teh directions, is also needed for the proof of showing poly has equalizers. But where possible, we use the normal polynomials in proofs, to be as general as possible, and avoiding the use of isSet which would add noise to the code.

% \todo{Is the terminology of Wild polynomial correct here, and is the name of Set Polynomial correct here as well?}


\section{Future work}
There are many areas for future work formalizing \textbf{Poly}. 

\subsection{Comonoids and small categories}
Finishing this isomorphism would be interesting - the existing attempt fails when translating the comonoid laws into category laws, mostly due to how difficult it is to work with the associativity proof. This suggests that a different formulation of comonoids, that does not need \texttt{transport}, might be fruitful.

\subsection{Bicomodules are parametric right adjoints}
Another construct covered in the poly-book is bicomodules, which relate to comonoids. Bicomodules have a practical use as data migration for databases \cite{bicomodulesBlog} as well as effect handlers. Therefore, implementing and using bicomodules should be of particular interest to software developers.

\subsection{Composition of squares between charts and lenses} \label{section:commutingSquaresCompose}
The commuting squares between charts and lenses have been defined as \mintinline{agda}{LensChartCommute}. Showing that these squares compose both horizontally and vertically is an important step in finalizing the formalization of the double category between charts and lenses. The proof that the map on positions of the squares composes is done, but the proof for map on directions remains undone, due to an explosion of subst's.

\subsection{Consistency of distributed data types}
There is some speculation that polynomial functors can be a good model of consistent distributed data storage, like Conflict-Free Replicated Data Types, so the categorical setting could be fruitful for proving the consistency of CRDTs.

\subsection{Other categorical properties}
The categorical constructs and properties formalized in this thesis comprise only a handful of all the properties that \textbf{Poly} exhibits. A natural extension is to formalize more constructs. For instance, the coequalizer, described informally in natural language in the poly-book, is an important construct as it shows that \textbf{Poly} has all small colimits. The poly-book contains plenty of more proofs and examples that could be formalized. 

\subsection{Dynamical systems}
This thesis implements several dynamical systems, but it is a large field. Systems come in various flavors: time-delayed, open, closed, chaotic, stable, unstable, discrete, continuous, stochastic, deterministic, hybrid, etc. It would be interesting to continue to explore this landscape while leveraging the properties of \textbf{Poly} in the dynamical systems. For instance, our use of mode-dependence, in which we believe lies the true potential of \textbf{Poly}, is minor. Implementing more types of dynamical systems could also lead to common patterns or problems arising, whose solutions could lead to a better ecosystem in using \textbf{Poly} for modeling dynamics.

\subsection{Interaction between different fields that Poly models}

\textbf{Poly} is good at modeling databases, dynamical systems, and information theory, three fields that are loosely connected. \textbf{Poly} could be a setting in which to deepen these connections. \textbf{Poly} and coalgebras on it have also shown some promise in deep learning.

\subsection{Building programming languages on top of Poly}
As we've seen, \textbf{Poly} is cartesian closed, meaning it can model the Simply-Typed Lambda Calculus. One could pursue the development of programming languages with the primitives of \textbf{Poly} that have access to the modeling power of the category natively. \textbf{Poly} also naturally models Turing machines and is suited to imperative programming - since the STLC is equivalent in expressive power to Universal Turing machines, \textbf{Poly} can also remain the setting where the equivalence between imperative and declarative programs is formally proved.
% \subsection{\textbf{Poly} in software engineering}
% An interesting research area is how \textbf{Poly} can be used in software engineering. Whether the tools provided by \textbf{Poly} can aid in the modeling of domain logic or state machines, or help software engineers in any other way. Perhaps mode dependence can add another dimension to state machines.


